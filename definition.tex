\section{Définition de la donnée}

\subsection{La donnée}

Pour pouvoir mener correctement cette étude, il convient de définir le
terme même de \emph{donnée}. Le petit Robert \cite{robert2007} propose
la définition suivante, applicable dans le domaine informatique :

\begin{quotation}
  « Représentation conventionnelle d'une information (fait, notion,
  ordre d'exécution) sous une forme (analogique ou digitale)
  permettant d'en faire le traitement automatique. »
\end{quotation}

Dans cette définition apparaissent plusieurs termes qui présentent un
grand intérêt pour notre étude.

Une donnée est une \emph{représentation}, ce qui explicite son
caractère immatériel : elle ne peut donc pas faire l'objet d'un droit
de propriété tel que définit dans le Code civil. En effet, la
propriété y est définie à l'article 544 comme « le droit de jouir et
disposer des choses [...] ». Un eventuel droit de propriété sur une
donnée serait plutôt à rapprocher de la propriété intellectuelle.

De plus, cette représentation est \emph{conventionnelle} : cela est en
effet indispensable pour pouvoir la communiquer à d'autres personnes.
Si cette convention est régie par un texte, telle qu'une norme, son
auteur dispose de droits sur le texte lui-même. Peut-il étendre ses
droits sur toutes les données formalisées selon cette convention ?

Enfin, une donnée représente une \emph{information}. Une information
est une mise en forme, le plus souvent verbale, de faits, ayant pour
finalité leur communication \cite{littreInfo} \cite{larousseInfo} Les
informations font l'objet d'un droit particulier, censé promouvoir
leur libre circulation. Si les données n'en sont qu'une
représentation, ne faudrait-il pas leur appliquer un droit similaire ?

Pour illustrer cette définition, considérons l'exemple suivant : on
trouve régulièrement dans les rues des bornes d'incendie à l'usage des
pompiers : c'est un fait. Lorsque les pompiers interviennent pour un
incendie, ils ont besoin de savoir s'il y a des bornes à
proximité. Pour acquérir cette connaissance, ils ont besoin d'accéder
à l'information correspondante. Pour garantir que les fournisseurs et
toutes les casernes aient la même compréhension de l'information, on
décide d'une convention de représentation : par exemple, on peut
spécifier la projection utilisée, le format de fichier, \dots{} En
définitive, l'information est transmise sous forme de donnée.


\subsection{Le créateur}

Puisque la donnée semble se rapprocher d'une œuvre, peut-être est-il
possible de lui attribuer un auteur ? Le tout premier article du Code
de la propriété intellectuelle (L111-1) définit simultanément l'auteur
et son rapport à l'œuvre :

« L'auteur d'une oeuvre de l'esprit jouit sur cette oeuvre, du seul
fait de sa création, d'un droit de propriété incorporelle exclusif et
opposable à tous. »

Dans cette définition, deux mots interdissent son application aux
données.  D'abord, le droit d'auteur protège des œuvres de
l'esprit. Or, la donnée est généralement créée par une mesure ou
l'application d'un traitement, qui sont réalisables par toute personne
(éventuellement disposant d'une formation ou de matériel), et ne
peuvent pas être attribuées à une personne en particulier. Cependant,
puisque la donnée est une représentation \emph{conventionnée}, le
texte énonçant cette convention a un auteur qui est protégé ; il en va
de même de la personne ayant énoncé un protocole de traitement ou
breveté un instrument de mesure. De même qu'un organisme de
normalisation tel que l'ISO ne peut prétendre à des droits sur les
matériaux se conformant à une norme, l'auteur des protocoles de mesure
ou de traitement ne peut pas s'attribuer un droit sur les données
produites par son procédé.

Deuxièmement, la définition ci-dessus suppose que les œuvres protégées
par le droit d'auteur aient été \emph{créées}. Or, il existe des
données ne pouvant pas être attribuées à un créateur, par exemple en
géograhie : qui a imaginé que les coordonnées géographiques de Paris
sont 48° 51′ 24″ Nord 2° 21′ 07″ Est et que la longueur de la Loire
est de 1006 km ? En ce qui concerne les données étant issues du
traitement par l'homme d'autres données, il est raisonnable de se
demander si ce traitement ne constitue pas la création d'une œuvre au
sens du droit d'auteur. En général, le traitement subi est spécifié de
manière rigoureuse, afin que le créateur et l'utilisateur de la donnée
en aient la même interprétation\footnote{Un exemple de traitement de
  données non spécifié :
  http://www.explainxkcd.com/wiki/index.php/1572:_xkcd_Survey}. Toute
personne disposant des mêmes données initiales et du protocole de
traitement est par conséquent en mesure de refabriquer exactement la
même donnée, ce qui est contraire au principe d'originalité des
créations intellectuelles. Par conséquent, il semble plus rigoureux de
protéger le créateur du protocole de traitement que la personne
l'ayant appliqué à des données.

Notons toutefois que, à la section 1 de la Circulaire du 14 février
1994 relative à la diffusion des données publiques, le Premier
ministre Édouard Balladur marque une distinction entre les données
brutes et traitées :

\begin{quotation}
  Une distinction doit être faite entre données brutes et données
  élaborées :
  \begin{itemize}
  \item les données brutes élémentaires, sans mise en forme
    originale, ne sont en principe la propriété de personne ;
  \item en revanche, la valeur ajoutée par l'administration est
    susceptible d'appropriation intellectuelle. Elle peut alors en
    céder l'usage dans les conditions prévues par la législation sur
    la propriété intellectuelle.
  \end{itemize}
\end{quotation}

On peut aussi trouver des exemples de données directement protégées
par le droit d'auteur, comme un livre numérique au format e-book,
l'enregistrement d'un concert du groupe à la mode, un didacticiel en
vidéo, \dots.


%%% Local Variables:
%%% mode: latex
%%% TeX-master: "main"
%%% End:
