\section{Définition de la donnée}

\subsection{La donnée}

Pour pouvoir mener correctement cette étude, il convient de définir le
terme même de \emph{donnée}. Le petit Robert \cite{robert2007} propose
la définition suivante, applicable dans le domaine informatique :

\begin{quotation}
  « Représentation conventionnelle d'une information (fait, notion,
  ordre d'exécution) sous une forme (analogique ou digitale)
  permettant d'en faire le traitement automatique. »
\end{quotation}

Dans cette définition apparaissent plusieurs termes qui présentent un
grand intérêt pour notre étude.

Une donnée est une \emph{représentation}, ce qui explicite son
caractère immatériel : elle ne peut donc pas faire l'objet d'un droit
de propriété tel que définit dans le Code civil \cite{ccivil}.

De plus, cette représentation est \emph{conventionnelle} : cela est en
effet indispensable pour pouvoir la communiquer à d'autres personnes ;
néanmoins, à supposer qu'il existe un auteur pour une donnée, une
convention laisse rarement à l'auteur une occasion de manifester
l'originalité qui lui confèrerait un droit sur son œuvre. % citer le
                                % code de la propritété
                                % intellectuelle.

Enfin, une donnée représente une \emph{information}. Une information
est une mise en forme, le plus souvent verbale, de faits, ayant pour
finalité leur communication. Les informations font l'objet d'un droit
particulier, censé promouvoir leur libre circulation. Si les données
n'en sont qu'une représentation, ne faudrait-il pas leur appliquer un
droit similaire ?

Pour illustrer cette définition, considérons l'exemple suivant : on
trouve régulièrement dans les rues des bornes d'incendie à l'usage des
pompiers : c'est un fait. Lorsque les pompiers interviennent pour un
incendie, ils ont besoin de savoir s'il y a des bornes à
proximité. Pour acquérir cette connaissance, ils ont besoin d'accéder
à l'information correspondante. Pour garantir

\subsection{L'auteur}





%%% Local Variables:
%%% mode: latex
%%% TeX-master: "main"
%%% End:
